\documentclass{article}

\usepackage{listings}
\usepackage{xcolor}

\definecolor{codegreen}{rgb}{0,0.6,0}
\definecolor{codegray}{rgb}{0.5,0.5,0.5}
\definecolor{codepurple}{rgb}{0.58,0,0.82}
\definecolor{backcolour}{rgb}{0.95,0.95,0.92}

\lstdefinestyle{mystyle}{
    backgroundcolor=\color{backcolour},   
    commentstyle=\color{codegreen},
    keywordstyle=\color{magenta},
    numberstyle=\tiny\color{codegray},
    stringstyle=\color{codepurple},
    basicstyle=\ttfamily\footnotesize,
    breakatwhitespace=false,         
    breaklines=true,                 
    captionpos=b,                    
    keepspaces=true,                 
    numbers=left,                    
    numbersep=5pt,                  
    showspaces=false,                
    showstringspaces=false,
    showtabs=false,                  
    tabsize=2
}

\lstset{style=mystyle}

\author{Anders Jacobsen, Dima Karaush}
\title{Optimization of Letter Frequencies - Take Two}

\begin{document}
% \lstset{language=Java}

\maketitle

\tableofcontents

\section{Test Enviroment}
System information
Java information
IDE info? 

\section{Introduction}
what are we working with
what approach do we take on the task
what results are we expecting? 

\section{Tools for Benchmarking}
Timer klassen
Benchmark timeren
Lavet metoder statiske og kalder dem ved benchmarking

\section{Benchmark for original}
What is going on in the program
As it is visible on Listing \ref{lst:originalmainmethod}, the original letter frequencies program uses a FileReader 
and a HashMap\textless Integer, Long\textgreater \space to read the file and safe the letter frequencies. It 
manipulates the Hashmap through the static tallyChars method. Then the program uses 
the other static method, print\_tally, to show the letters alligned with their 
frequency in the file. 


\begin{lstlisting}[caption={The main method of the original Letter Frequncies program without optimizations},label={lst:originalmainmethod},language=Java]
public static void main(String[] args) throws FileNotFoundException, IOException {

    String filePath = "src/main/resources/FoundationSeries.txt";

    Reader reader = new FileReader(filePath);
    Map<Integer, Long> freq = new HashMap<>();

    tallyChars(reader, freq);

    print_tally(freq);
}
\end{lstlisting}

code snippets of benchmark points
How did we make the benchmark method
What can be optimized

\section{Optimization}
What changes do we make to the program
Show snippets of changes

\section{Benchmark for Optimized}
show the results of the benchmark

\section{Benchmark Comparison}
Compare the original vs the optimized times
show tons of graphs and charts

\section{Conclusion}
Sum it up 
talk about most remarkable changes.


\end{document}