\documentclass{article}

\usepackage{xurl}
\usepackage{listings}

\usepackage{pgfplots}
\pgfplotsset{compat=1.8}
\usepgfplotslibrary{statistics}


\author{Anders Jacobsen, Dima Karaush}
\title{Exploration and Presentation - Assignment 3}


\begin{document}
\maketitle

\newpage
\tableofcontents

\newpage
\section{Intro}
We are using the project Letter Frequencies downloaded 
from \url{https://github.com/CPHBusinessSoftUFO/letterfrequencies} 
in this paper. The entire project with all files can be found here 
\url{https://github.com/Cosby1992/CPH-business-assingments/tree/master/Data%20Science/Assignment_3_optimization}.

\section{System Information}
\begin{tabular}{ l l }
OS Name                 & Microsoft Windows 10 Pro \\ 
OS Version              & 10.0.19042 N/A Build 19042 \\  
System Type             & x64-based PC \\
Processor(s)            & Intel® Core™ i7-10700KF Processor, 16M Cache, up to 5.10 GHz \\
BIOS Version            & American Megatrends Inc. 1.10, 21-05-2020 \\
Total Physical Memory   & 32.688 MB \\ 
Disc(s)                 & Force Series™ MP510 980GB M.2 SSD (up to 3480MB/sec read)
\end{tabular}

\section{Enviroment}
\begin{tabular}{ l l }
IDE                     & Visual Studio Code \\ 
IDE version             & 1.55.2 x64 \\  
Language                & Java \\
Language Version        & 15 \\
\end{tabular}

\section{Analysis - Task 1}
\begin{enumerate}
    \item Find a point in your program that can be optimized (for speed), 
    for ex-ample by using a profiler
    \item Make a measurement of the point to optimize, for example by 
    running a number of times, and calculating the mean and standard 
    deviation (see the paper from Sestoft)
    \item If you work on the letter frequencies program, make it at least 50\% faster
\end{enumerate}

\subsection{Find a point in your program that can be optimized}
We did not use a profiler. The reason is that we could not get 
the profiler to run in Visual Studio Code. We have however located 
multiple points in the program that can be optimized. Since we are 
reading a relatively big text-file, we believe it is here we can 
optimize the most. We took multiple steps to make the program faster
as shown in the list below:
\begin{enumerate}
    \item Upgrading the FileReader to a BufferedReader.
    \item Replacing the ``HashMap\textless Integer, Long\textgreater`` with a ``int[]``
    \item Limiting the while loop to only saving letters A-Z
\end{enumerate}
We've decided to measure the two methods in the program ``tallyChars()'' and ``print\_tally()''. 
We measure them together as a sequence to see the difference execution time on all our optimizations. 

\subsection{Bottlenecks}
\begin{enumerate}
  \item A bottleneck in this program was the use of a FileReader instead of a BufferedReader. 
  A buffer when reading a file is a huge advantage and can significantly improve reading times of the file.
  \item Another bottleneck in this program was that there were no checks on weather it was needed to count 
  the frequency of the data. It should be limited to only count characters a-z.
  \item The use of a HashMap and LinkedHashMap instead of primitive data type array. It is much faster to 
  write a number to a now location in an Array than to add to values to a HashMap or LinkedHashMap.
\end{enumerate}
A bottleneck in this program was the use of a FileReader instead of a BufferedReader. 
A buffer when reading a file is a huge advantage and can significantly improve reading times of the file. 


\subsection{Make a measurement of the point to optimize}
To make measurements we first had to create a timer that supports our 
measurements. The timer class can be found in 
``letterfrequencies/src/main/java/cphbusiness/ufo/letterfrequencies/Timer.java''.

\subsubsection*{Procedure}
We've decided to make multiple measurements in order to create a realistic image 
of the run-times of both the optimized and non-optimized classes. The way we do this
is by creating a loop and then run the two methods for a specified amount of iterations
and writing each iterations run-time to a CSV file for the analysis. 

We've chosen to use 500 continuous measurements to calculate the statistics afterwords.


\subsection{Make it at least 50\% faster}
The task was to make the program 50\% faster, and that has been achieved. 
Below are some tables and statistics along with a box-plot descriping our results.

\begin{tabular}{ l l l }
                & Normal        & Optimized \\ 
    Mean        & 46,562ms      & 23,232ms \\  
    Std. dev    & 3,351510293   & 0,571690178 \\
    Min         & 40ms          & 23ms \\
    1. Quatile  & 46ms          & 23ms \\
    Median      & 47ms          & 23ms \\
    3. Quartile & 47ms          & 23ms \\
    Max         & 99ms          & 26ms    
\end{tabular}

 
From this data it is possible to create a box-plot to visualize the difference. 
\newline

\begin{tikzpicture}
    \begin{axis}
      [
      ytick={1,2},
      yticklabels={Normal, Optimized},
      ]
      \addplot+[
      boxplot prepared={
        median=47,
        upper quartile=47,
        lower quartile=46,
        upper whisker=99,
        lower whisker=40
      },
      ] coordinates {};
      \addplot+[
      boxplot prepared={
        median=23,
        upper quartile=23,
        lower quartile=23,
        upper whisker=26,
        lower whisker=23
      },
      ] coordinates {};
    \end{axis}
  \end{tikzpicture}

As it can be seen in the box-plot, there is a few outliers that could be removed. Esspecially from the Normal plot. 
With the ouliers removed, the new boxplot looks like the plot below: 
\newline
\begin{tikzpicture}
    \begin{axis}
      [
      ytick={1,2},
      yticklabels={Normal, Optimized},
      ]
      \addplot+[
      boxplot prepared={
        median=47,
        upper quartile=47,
        lower quartile=46,
        upper whisker=50,
        lower whisker=43
      },
      ] coordinates {};
      \addplot+[
      boxplot prepared={
        median=23,
        upper quartile=23,
        lower quartile=23,
        upper whisker=24,
        lower whisker=23
      },
      ] coordinates {};
    \end{axis}
  \end{tikzpicture}

It is now possible to see that not only did we improve a whole lot on the execution-time,
but we have also stabilized the system a whole lot. This is visible from observing the 
whiskers on each entry in the box-plot. 
As it is visible from the results table, the results have improved by a whole lot, 
but how much exactly? This depends on what you are measuring, we've decided that 
two of the numbers we can use to illustrate the improvements are the Mean and the 
Median. Therefore we have calculated how much the run-times have improved in percent. 
The results can be seen below: \newline

\begin{tabular}{ l l }
Mean improvement & 50,11\% \\
Median improvement & 51,06\% \\ 
\end{tabular}

\subsubsection*{Results}
In conclution we have improved the program by approx. 50\%-51\%. 
But what is also worth noticing is the stabilization of the execution-times
with our optimized solution. 

\section{Steps to reproduce}
Go to \url{https://github.com/Cosby1992/CPH-business-assingments/tree/master/Data%20Science/Assignment_3_optimization} 
and follow the instructions from the README.md

\end{document}