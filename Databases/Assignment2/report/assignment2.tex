\documentclass{article}

\usepackage{cite}


\title{Databases for Developers - Assignment 2}
\author{Anders Jacobsen, Dima Karaush}

\begin{document}
\maketitle

\section{Task 1 - investigation}
Produce a small writeup (around 300 words) answering the following questions.
\begin{enumerate}
\item What is point of NoSQL databases?
\item What is the CAP theorem?
\item What are ideal use cases of HBase?
\end{enumerate}

\subsection{What is the point of NoSQL databases}
The point of NoSQL databases is to fill the holes where SQL databases does not. 
We handle data in many different ways, so it is just logical that we should have 
multiple tools for storing that data. Some pros and cons can be found below. 
\subsubsection*{Pros}
\begin{itemize}
    \item Performance
    \item Scalability
    \item Flexibility
    \item Data Models
\end{itemize}

\subsubsection*{Cons}
\begin{itemize}
    \item Not mature
    \item Requires multiple databases
    \item Huge databases
\end{itemize}

But the above information depends heavily on witch NoSQL database you choose. There are many!

\subsection{What is the CAP theorem}
In theoretical computer science, the CAP theorem, also named Brewer's theorem after computer 
scientist Eric Brewer, states that it is impossible for a distributed data store to 
simultaneously provide more than two out of the following three guarantees\cite{wiki_CAP}:
\begin{itemize}
    \item Consistency: Every read receives the most recent write or an error
    \item Availability: Every request receives a (non-error) response, without the guarantee 
    that it contains the most recent write
    \item Partition tolerance: The system continues to operate despite an arbitrary number of 
    messages being dropped (or delayed) by the network between nodes
\end{itemize}

\subsection{What are ideal use cases of HBase}
HBase uses Google's BigTable paper as blueprint, HBase is build on Hadoop and the Hadoop 
Distributed File System (HDFS) and designed for scaling horizontally on clusters of 
commodity hardware. HBase makes strong consistency guarantees and features tables with rows 
and columns - with should make SQL fans feel right at home. Out-of-the-box support for 
versioning and compression\cite{sevendbsinsevenweeks}.


\section{Task 2 - Bloom filters}
Bloom filters are used in hbase as an incredible optimization. Solve the following.
\begin{enumerate}
    \item What is a bloom filter?
    \item What is an advantage of bloom filters over hash tables?
    \item What is a disadvantage of bloom filters?
    \item Using your language of choice, implement a bloom filter with add and check functions. The backing bit-array can simply be a long (64 bit integer)
    \item If you are to store one million ASCII strings with an average size of 10 characters in a hash set, what would be the approximate space consumption?
    \item The following equation gives the required number of bits of space per inserted key, where E is the false positive rate. \( b = 1.44log_2(1/\xi)\)
    \item How many bits per element are required for a 1\% false positive rate?
    \item How many bits per element are required for a 5\% false positive rate?
    \item If you are to store one million ASCII strings with an average size of 10 characters in a bloom filter, what would be the approximate space consumption, given an allowed false positive rate of 5\%?.

\end{enumerate}

\subsection{What is a bloom filter}
A Bloom filter is a data structure designed to tell you, rapidly and memory-efficiently, 
whether an element is present in a set. The price paid for this efficiency is that a Bloom 
filter is a probabilistic data structure: it tells us that the element either definitely 
is not in the set or may be in the set\cite{bloomfilterbyexample}.

\subsection{What is an advantage of bloom filters over hash tables}
Bloom filters is more space efficient. Since it does not store the actual object.

\subsection{What is a disadvantage of bloom filters}
The bloom filter is a  probabilistic data structure, this means that the result has a small 
risk of returning as a "false positive". This could mean that you get a positive result 
for an object that does not exist.

\subsection{Using your language of choice, implement a bloom filter with add and check functions}
We have made a very simple exapmle of a bloom filter in python, it can be found in the folder "bloom\_filter".

\subsection{If you are to store one million ASCII strings with an average size of 10 characters in a hash set, what would be the approximate space consumption}

\[ One ASCII character = 8 bits \] 
\newline
\[ String lenght 10 = 80 bits \]
\newline
\[ Times a million = 80.000.000 bits \]
\newline
\[ 80.000.000 bits = 10 Mb (Megabytes) \]

For each entry the HashSet normally takes an empty space of equal 
size of the object as well as a pointer to the next entry.  
In Java the standart HashSet implementations memory consumption 
can be defined as \[ 32 * SIZE + 4 * CAPACITY \]

The default Capacity of a HashSet implementation is 16

Therefore \[ 32*1000000+4*16 = 32.000.064 bytes = 30.5176 Mb \]

So the approximate memory consumtion for a HashSet in Java 
for storing 1.000.000 10 letter string would be 30.5176 Mb

\subsection{The following equation gives the required number of bits of space per inserted key, where E is the false positive rate}
\label{section:equation}
\[ b = 1.44log_2(1/\xi)\]

\subsection{How many bits per element are required for a 1\% false positive rate}
We insert the probability (procent as decimal) in the equation from section \ref{section:equation}:
\[ b = 1.44log_2(1/0.01) = 9.56715291328\]

We can now conclude that we need approx. 9.57 bits of space pr. inserted key, to get a 1\% false positive rate.


\subsection{How many bits per element are required for a 5\% false positive rate}
\label{section:bitspritem}
We insert the probability (procent as decimal) in the equation from section \ref{section:equation}:
\[ b = 1.44log_2(1/0.05) = 6.22357645664\]

We can now conclude that we need approx. 6.22 bits of space pr. inserted key, to get a 5\% false positive rate.

\subsection{If you are to store one million ASCII strings with an average size of 10 characters in a bloom filter, what would be the approximate space consumption, given an allowed false positive rate of 5\%}
We insert the space consumtion from section \ref{section:bitspritem}:
\[ 6.22357645664*1000000 = 6223576.45664\]

The approx. space consumtion would therefore be 6223576.46 bits that equals approx. 759.71 Kilobytes. 

\section{Task 3 - Huffman coding}
HBase internally uses a compression that is a combination of LZ77 and Huffman Coding.
\begin{enumerate}
    \item Generate Huffmann Code (and draw the Huffmann Tree) based on the following string: “beebs beepps!!!!! their eerie ears hear pears”
    \item How many bits is the compressed string? How many bits is the raw ASCII string?
    \item Compress “pete is here” with the Huffmann tree from before.
    \item Write your own 10 word sentence. Generate the Huffmann Code (a new Huffmann Tree), and write a new compressed message (ie. in binary). Swap with one of your fellow students, and decompress each other’s message.

\end{enumerate}



\bibliography{bibliography}{}
\bibliographystyle{plain}



\end{document}
