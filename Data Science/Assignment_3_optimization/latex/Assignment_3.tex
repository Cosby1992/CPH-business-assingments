\documentclass{article}

\usepackage{xurl}
\usepackage{listings}


\author{Anders Jacobsen, Dima Karaush}
\title{Exploration and Presentation - Assignment 3}


\begin{document}
\maketitle

\newpage
\tableofcontents

\newpage
\section*{Intro}
We are using the project Letter Frequencies downloaded 
from \url{https://github.com/CPHBusinessSoftUFO/letterfrequencies} 
in this paper. 

\section{Task 1}
\begin{enumerate}
    \item Find a point in your program that can be optimized (for speed), 
    for ex-ample by using a profiler
    \item Make a measurement of the point to optimize, for example by 
    running a number of times, and calculating the mean and standard 
    deviation (see the paper from Sestoft)
    \item If you work on the letter frequencies program, make it at least 50\% faster
\end{enumerate}

\subsection{Find a point in your program that can be optimized}
We did not use a profiler. The reason is that we could not get 
the profiler to run in Visual Studio Code. We have however located 
multiple points in the program that can be optimized. Since we are 
reading a relatively big text-file, we believe it is here we can 
optimize the most. We took multiple steps to make the program faster
as shown in the list below:
\begin{enumerate}
    \item Upgrading the FileReader to a BufferedReader.
    \item Replacing the ``HashMap\textless Integer, Long\textgreater`` with a ``int[]``
    \item Limiting the while loop to only saving letters A-Z
\end{enumerate}
We've decided to measure the two methods in the program ``tallyChars()'' and ``print\_tally()''. 
We measure them together as a sequence to see the difference execution time on all our optimizations.  

\subsection{Make a measurement of the point to optimize}
To make measurements we first had to create a timer that supports our 
measurements. The timer class can be found in 
``letterfrequencies/src/main/java/cphbusiness/ufo/letterfrequencies/Timer.java''.

\subsubsection*{Procedure}
We've decided to make multiple measurements in order to create a realistic image 
of the run-times of both the optimized and non-optimized classes. The way we do this
is by creating a loop and then run the two methods for a specified amount of iterations
and writing each iterations run-time to a CSV file for the analysis. 


\subsection{Make it at least 50\% faster}

\subsubsection*{Results}




\end{document}